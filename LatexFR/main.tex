%\title{Mon CV}
%
% tccv (two columns curriculum vitae) is a LaTeX class inspired by
% the template found at latextemplates.com by Alessandro Plasmati.
%
% Create by Nicola Fontana, the original files can be downloaded from:
% http://dev.entidi.com/p/tccv/
%
% Modified by Zhipeng JIANG
%
\documentclass{tccv}
\usepackage[utf8]{inputenc}

\begin{document}

\part{Zhipeng JIANG}

\section{Expérience professionnelle}

\begin{eventlist}

\item{Juin. 2013 -- Sept. 2013}
     {Reportlinker.com, Lyon France}
     {Front-end Développment Stage}

- RESTful APIs développement pour ElasticaSearch, Solr et MongoDB avec le framework Symfony2. 
\newline
- Atoum functional/unit tests et code refactoring avec le Decorateur design pattern.
\newline
- Android développment avec l'achitecture Android + SQLite3 + RESTful APIs(Symfony2) + MongoDB/ElasticSearch. ActionBarSherlock, ViewPagerIndicator et autre open-source libraries sont intégrés dans cette application.

\item{Sept. 2012 -- Mai. 2013}
     {Loria, Nancy France}
     {AR.Drone Lab Développment}

- Détection et reconnaissance vocale développment. Avec l'API de reconnaissance vocale de Kinect, Drone peut être contrôlé par des commandes vocales prédéfinies.
\newline
- AR.Drone 2D positionnement développment avec l'API de Kinect. Connaissant les positions des deux KINECTS et deux directions relatives, la position du drone peut être calculée à l'intérieur d'un triangle dans l'espace 2D.
\newline
- Socket développment entre Kinect(C\#) and Drone control platform(Java).


\item{Juillet. 2012 -- Oct. 2012}
     {StanDroid, Nancy France}
     {Open-source Project Développment}

Je suis le propriétaire et développeur de normes StanDroid, une open-source Android application, basé sur LBS(Location Based Service), fournissant l'informations pour le système de transport en commun Nancy. Ce projet inclut la technologie de text mining de base et l'optimisation des performances des bases de données. En plus, StanDroid compte plus de 1500 utilisateurs et est classé 4.6/5.0 dans Google Play. La prochaine version sera bientôt disponible.

\item{Dec. 2011 -- Fev. 2012}
     {Ericsson Research and Development, Shanghai}
     {Lab APIs Développment Stage}

- Test de Ericsson Research Lab API pour la plateforme Android, notamment le Voice Recognition API, le Voice Speech API et le Location Positioning API.
\newline
- Tester et utiliser Wikipedia API pour développer une application qui permet de chercher vocale, lecture les résultat et sélectioner la version linguistique du document en fonction API LBS.

\end{eventlist}

\personal
    [http://jesusjzp.github.io]
    {03 Rue du Général Hoche\newline 54000 -- Nancy, France}
    {+33 7 70 39 37 74}
    {zhipeng.jiang7@etu.univ-lorraine.fr}

\section{Formation}

\begin{yearlist}

\item[Ingénieur]{2012 -- présent}
     {Information et Systeme}
     {École des Mines de Nancy, FR}

\item[Bachelor(GPA:87.4/100)]{2009 -- 2012}
     {Software Engineering}
     {Shanghai Jiao Tong University}

\end{yearlist}

\section{Prix et Bourse}

\begin{yearlist}

\item{Avr. 2012}
     {Top 30 (Chine)}
     {Compétition Internationale de Microsoft Imagine Cup Software Design.}
     
\item{Jan. 2012}
     {La 7ème place (parmi 176)}
     {Compétition Nationale Ericsson Concours Créatif.}

\end{yearlist}

\section{Extracurricular}

\begin{factlist}
\item{Siyuan NGO}{Providing free English courses to the children in the poorest area of China.}
\item{Expo 2010}{Volunteering as a tourist guide.}
\end{factlist}

\section{Communication skills}

\begin{factlist}
\item{English}{Oral: fluent -- Written: good}
\item{French}{Oral: fair}
\item{Chinese}{Native speaker}
\end{factlist}

\section{Software skills}

\begin{factlist}

\item{Good level}
     {Symfony2, PHP, Android development, Java, MySQL, Python, Git, Linux}

\item{Intermediate}
     {C/C++, C\#, MongoDB, Jekyll, Markdown, Ruby, Perl, Lisp, LaTeX}

\end{factlist}

\end{document}
